\documentclass{article}
\usepackage{geometry}
\geometry{a4paper, margin=1in}
\usepackage{amsmath}
\usepackage{listings}
\usepackage{color}
\usepackage{xcolor}
\usepackage{kotex}

\definecolor{codegray}{gray}{0.9}
\definecolor{codeblue}{rgb}{0.0, 0.0, 1.0}

\lstdefinestyle{mystyle}{
    backgroundcolor=\color{codegray},
    basicstyle=\ttfamily\footnotesize,
    keywordstyle=\color{codeblue},
    breakatwhitespace=false,
    breaklines=true,
    captionpos=b,
    keepspaces=true,
    numbers=left,
    numbersep=5pt,
    showspaces=false,
    showstringspaces=false,
    showtabs=false,
    tabsize=2,
    language=Python
}

\lstset{style=mystyle}

\title{Flutter 기본 위젯 정리}
\author{}
\date{}

\begin{document}
\maketitle

\section*{레이아웃}
\subsection*{Container}
회화, 위치 지정 및 크기 조정을 위한 일반적인 위젯을 결합한 편리한 위젯입니다.
\textbf{예시:}
\begin{lstlisting}
Container(
    width: 100,
    height: 100,
    color: Colors.blue
)
\end{lstlisting}

\subsection*{Row}
자식 위젯을 가로로 배열하는 위젯입니다.
\textbf{예시:}
\begin{lstlisting}
Row(
    children: <Widget>[
        Text('Hello'),
        Text('World')
    ]
)
\end{lstlisting}

\subsection*{Column}
자식 위젯을 세로로 배열하는 위젯입니다.
\textbf{예시:}
\begin{lstlisting}
Column(
    children: <Widget>[
        Text('Hello'),
        Text('World')
    ]
)
\end{lstlisting}

\subsection*{Stack}
자식 위젯을 서로 위에 쌓아 표시하는 위젯입니다.
\textbf{예시:}
\begin{lstlisting}
Stack(
    children: <Widget>[
        Text('First'),
        Text('Second')
    ]
)
\end{lstlisting}

\subsection*{SizedBox}
고정된 크기를 가지는 상자를 만드는 위젯입니다.
\textbf{예시:}
\begin{lstlisting}
SizedBox(
    width: 100,
    height: 100
)
\end{lstlisting}

\subsection*{Padding}
자식 위젯에 패딩을 추가하는 위젯입니다.
\textbf{예시:}
\begin{lstlisting}
Padding(
    padding: EdgeInsets.all(16.0),
    child: Text('Hello')
)
\end{lstlisting}

\subsection*{Center}
자식 위젯을 중앙에 배치하는 위젯입니다.
\textbf{예시:}
\begin{lstlisting}
Center(
    child: Text('Hello')
)
\end{lstlisting}

\subsection*{Align}
자식 위젯을 정렬하는 위젯입니다.
\textbf{예시:}
\begin{lstlisting}
Align(
    alignment: Alignment.topRight,
    child: Text('Hello')
)
\end{lstlisting}

\subsection*{Expanded}
자식 위젯을 가능한 한 많이 확장시키는 위젯입니다.
\textbf{예시:}
\begin{lstlisting}
Expanded(
    child: Text('Hello')
)
\end{lstlisting}

\subsection*{Flexible}
자식 위젯을 유연하게 확장시키는 위젯입니다.
\textbf{예시:}
\begin{lstlisting}
Flexible(
    child: Text('Hello')
)
\end{lstlisting}

\subsection*{Spacer}
Row나 Column에서 빈 공간을 차지하는 위젯입니다.
\textbf{예시:}
\begin{lstlisting}
Spacer()
\end{lstlisting}

\section*{텍스트 및 이미지}
\subsection*{Text}
단일 스타일로 텍스트를 표시하는 위젯입니다.
\textbf{예시:}
\begin{lstlisting}
Text(
    'Hello, World!',
    style: TextStyle(fontSize: 20)
)
\end{lstlisting}

\subsection*{RichText}
여러 텍스트 스타일을 포함할 수 있는 텍스트 위젯입니다.
\textbf{예시:}
\begin{lstlisting}
RichText(
    text: TextSpan(
        children: [
            TextSpan(text: 'Hello '),
            TextSpan(text: 'World')
        ]
    )
)
\end{lstlisting}

\subsection*{Image}
이미지를 표시하는 위젯입니다.
\textbf{예시:}
\begin{lstlisting}
Image.network(
    'https://flutter.dev/assets/homepage/carousel/slide_1-bg-2x-3d45e5c97d6a2a80e0413d1c27d9f5cf9b9c39e426a730ec5a58c7e3a9f6a4c7.png'
)
\end{lstlisting}

\subsection*{Icon}
아이콘 글리프로 그려진 그래픽 아이콘 위젯입니다.
\textbf{예시:}
\begin{lstlisting}
Icon(
    Icons.favorite,
    color: Colors.pink,
    size: 24.0
)
\end{lstlisting}

\section*{입력}
\subsection*{Button}
다양한 유형의 버튼이 있는 머티리얼 디자인 버튼입니다: ElevatedButton, TextButton, OutlinedButton.
\textbf{예시:}
\begin{lstlisting}
ElevatedButton(
    onPressed: () {},
    child: Text('Click Me')
)
\end{lstlisting}

\subsection*{TextField}
사용자로부터 텍스트 입력을 받을 수 있는 위젯입니다.
\textbf{예시:}
\begin{lstlisting}
TextField()
\end{lstlisting}

\subsection*{Checkbox}
사용자가 선택할 수 있는 체크박스 위젯입니다.
\textbf{예시:}
\begin{lstlisting}
Checkbox(
    value: true,
    onChanged: (bool? newValue) {}
)
\end{lstlisting}

\subsection*{Radio}
라디오 버튼 위젯입니다.
\textbf{예시:}
\begin{lstlisting}
Radio(
    value: 1,
    groupValue: 1,
    onChanged: (int? newValue) {}
)
\end{lstlisting}

\subsection*{Switch}
켜기/끄기 스위치 위젯입니다.
\textbf{예시:}
\begin{lstlisting}
Switch(
    value: true,
    onChanged: (bool newValue) {}
)
\end{lstlisting}

\subsection*{Slider}
사용자가 값을 선택할 수 있는 슬라이더 위젯입니다.
\textbf{예시:}
\begin{lstlisting}
Slider(
    value: 50,
    min: 0,
    max: 100,
    onChanged: (double newValue) {}
)
\end{lstlisting}

\subsection*{DatePicker}
날짜를 선택할 수 있는 위젯입니다.
\textbf{예시:}
\begin{lstlisting}
showDatePicker(
    context: context,
    initialDate: DateTime.now(),
    firstDate: DateTime(2000),
    lastDate: DateTime(2100)
)
\end{lstlisting}

\subsection*{TimePicker}
시간을 선택할 수 있는 위젯입니다.
\textbf{예시:}
\begin{lstlisting}
showTimePicker(
    context: context,
    initialTime: TimeOfDay.now()
)
\end{lstlisting}

\subsection*{DropdownButton}
드롭다운 목록을 표시하는 버튼 위젯입니다.
\textbf{예시:}
\begin{lstlisting}
DropdownButton(
    items: [
        DropdownMenuItem(child: Text('One')),
        DropdownMenuItem(child: Text('Two'))
    ],
    onChanged: (value) {}
)
\end{lstlisting}

\subsection*{PopupMenuButton}
팝업 메뉴를 표시하는 버튼 위젯입니다.
\textbf{예시:}
\begin{lstlisting}
PopupMenuButton(
    itemBuilder: (context) => [
        PopupMenuItem(child: Text('One')),
        PopupMenuItem(child: Text('Two'))
    ]
)
\end{lstlisting}

\section*{내비게이션 및 구조}
\subsection*{Scaffold}
기본 머티리얼 디자인 시각적 레이아웃 구조를 구현합니다.
\textbf{예시:}
\begin{lstlisting}
Scaffold(
    appBar: AppBar(title: Text('App')),
    body: Center(child: Text('Hello, world!'))
)
\end{lstlisting}

\subsection*{AppBar}
머티리얼 디자인 애플리케이션의 상단 앱바입니다.
\textbf{예시:}
\begin{lstlisting}
AppBar(
    title: Text('Title')
)
\end{lstlisting}

\subsection*{BottomNavigationBar}
앱 하단에 네비게이션 바를 표시하는 위젯입니다.
\textbf{예시:}
\begin{lstlisting}
BottomNavigationBar(
    items: [
        BottomNavigationBarItem(icon: Icon(Icons.home), label: 'Home'),
        BottomNavigationBarItem(icon: Icon(Icons.settings), label: 'Settings')
    ]
)
\end{lstlisting}

\subsection*{Drawer}
앱의 왼쪽에 표시되는 드로어 메뉴 위젯입니다.
\textbf{예시:}
\begin{lstlisting}
Drawer(
    child: ListView(
        children: <Widget>[
            ListTile(title: Text('Item 1')),
            ListTile(title: Text('Item 2'))
        ]
    )
)
\end{lstlisting}

\subsection*{TabBar}
탭을 표시하는 위젯입니다.
\textbf{예시:}
\begin{lstlisting}
TabBar(
    tabs: [
        Tab(text: 'Tab 1'),
        Tab(text: 'Tab 2')
    ]
)
\end{lstlisting}

\subsection*{TabBarView}
탭에 따라 다른 콘텐츠를 표시하는 위젯입니다.
\textbf{예시:}
\begin{lstlisting}
TabBarView(
    children: [
        Text('Content 1'),
        Text('Content 2')
    ]
)
\end{lstlisting}

\subsection*{FloatingActionButton}
앱에서 플로팅 액션 버튼을 표시하는 위젯입니다.
\textbf{예시:}
\begin{lstlisting}
FloatingActionButton(
    onPressed: () {},
    child: Icon(Icons.add)
)
\end{lstlisting}

\section*{다이얼로그 및 팝업}
\subsection*{SnackBar}
일시적인 메시지를 표시하는 위젯입니다.
\textbf{예시:}
\begin{lstlisting}
ScaffoldMessenger.of(context).showSnackBar(
    SnackBar(content: Text('Hello'))
)
\end{lstlisting}

\subsection*{Dialog}
대화 상자를 표시하는 위젯입니다.
\textbf{예시:}
\begin{lstlisting}
showDialog(
    context: context,
    builder: (context) => AlertDialog(
        title: Text('Title'),
        content: Text('Content'),
        actions: [
            TextButton(
                onPressed: () {},
                child: Text('OK')
            )
        ]
    )
)
\end{lstlisting}

\subsection*{AlertDialog}
경고 대화 상자를 표시하는 위젯입니다.
\textbf{예시:}
\begin{lstlisting}
showDialog(
    context: context,
    builder: (context) => AlertDialog(
        title: Text('Alert'),
        content: Text('This is an alert dialog'),
        actions: [
            TextButton(
                onPressed: () {
                    Navigator.of(context).pop();
                },
                child: Text('OK')
            )
        ]
    )
)
\end{lstlisting}

\end{document}
